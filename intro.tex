\section{Введение}

В проектах «Компилятор формул» и «Интерпретатор арифметических
выражений» необходимо было расширить входную грамматику стекового компилятора операцией возведения в степень с указанными приоритетом и ассоциативностью, а также компилировать полученные формулы и вычислять соответствующие численные выражения. Реализация указанной модификации требует представления о формальных языках и грамматиках, об индуктивных функциях и простейших контейнерах данных. Так как в рамках учебного курса <<Алгоритмы и структуры данных>> проект должен быть реализован на языке Ruby, необходимы также базовые знания данного языка, как и знание основ объектно-ориентированного
программирования в целом~\cite{ruby}. 

Задачей проекта «Выпуклая оболочка»\cite{convex} является индуктивное 
перевычисление выпуклой оболочки последовательно поступающих точек плоскости и таких её
характеристик, как периметр и площадь. Целью данной работы является
определение расстояния от заданной прямой до выпуклой оболочки; вычисление радиуса максимального круга с центром в заданной точке, содержащегося в выпуклой оболочке. Решение этой задачи требует знания теории индуктивных функций, основ аналитической геометрии и векторной алгебры, а также языка Ruby~\cite{ruby}. Для наглядного изображения реализации задачи необходимы навыки работы с библиотекой графического интерфейса Tk~\cite{tk}. 

Проект «Изображение проекции полиэдра»~\cite{polyedr}~--- пример
классической задачи, для успешного решения которой необходимо знакомство с
основами вычислительной геометрии. Задачей, решаемой в данной работе, является
модификация эталонного проекта с целью определения суммы длин рёбер, середина и оба из концов которых~--- точки, проекция которых находится строго вне квадрата единичной площади с центром в начале координат и сторонами, параллельными координатным осям. Для этого необходимы хорошее понимание ряда разделов 
аналитической геометрии и векторной алгебры, основ объектно-ориентированного
программирования и языка Ruby. 
