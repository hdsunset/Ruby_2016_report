\section{Приложение к пункту 4.2}

В участки кода программы, отвечающие за удаление освещённых рёбер и добавление новых, были внесены следующие изменения:

При удалении или добавлении рёбер с некоторой вероятностью придётся пересчитывать радиус
  вписанной окружности. На самом этапе удаления ему будет временно присваиваться значение
    \verb|-1.0|. Далее искомый радиус будет вычисляться в зависимости от ситуации.


Учёт удаления ребра, соединяющего конец и начало дека:
\begin{lstlisting}
if (@@R - @@center.dist_to_line(@points.last,@points.first))<EPS
   @@R = -1.0
end
\end{lstlisting}


 Удаление освещённых рёбер из начала дека:
\begin{lstlisting}
@@R = -1.0 if @@R == @@center.dist_to_line(p, @points.first)
\end{lstlisting}

Удаление освещённых рёбер из конца дека:
\begin{lstlisting}
@@R = -1.0 if @@R == @@center.dist_to_line(p, @points.last)
\end{lstlisting}

В случае, если надо переопределить радиус вписанной окружности, мы также ищем минимальное
 из расстояний от центра до каждого ребра.

\begin{lstlisting}
if @@R == -1.0
  @@R = @@center.dist_to_line(@points.last, @points.first)
  @points.size.times do
    if @@R > @@center.dist_to_line(@points.last, @points.first)
    end
    @@R = @@center.dist_to_line(@points.last, @points.first)
    @points.push_last(@points.pop_first)
  end
else
 @@R=[@@R,@@center.dist_to_line(@points.last, @points.first)].min
 @points.push_last(@points.pop_first)
 @@R=[@@R,@@center.dist_to_line(@points.last, @points.first)].min
end
\end{lstlisting}
